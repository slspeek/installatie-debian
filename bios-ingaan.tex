\section{Algemene aanwijzingen}
\subsection{Typografische conventies}
Om te verduidelijken wat de bedoeling van een bepaald stukje tekst is, hanteren we typografi\"{e}en:
\begin{table}[H]
\begin{tabular}{| l | l |}
	\hline
	\textbf{Bold face} & voor invoer en optie namen van de Debian Installer\\
	\hline
	\texttt{Typewriter} & voor commando's voor de opdrachtregel en sneltoetsen\\
	\hline
\end{tabular}
\end{table}

\subsection{BIOS instellingen}
Om een installatie goed te kunnen uitvoeren is het meestal nodig de BIOS van de te installeren machine in te gaan. De toetsaanslagen om dit te bereiken verschillen per merk en model computer. Hieronder een overzicht per merk.
\begin{table}[H]
	\begin{tabular}{| l | l |}
	\hline 
	\textbf{Merk} & \textbf{Toetsen} \\
	\hline
	Compaq & \texttt{F10}, \texttt{F1}, \texttt{F2} of \texttt{Delete}\\
	\hline
	Dell & \texttt{F2} \\
	\hline
	Fujitsu & \texttt{F2} \\
	\hline
	Hewlet Packert &  \texttt{F1}, \texttt{F10}, \texttt{F11} \\
	\hline
	IBM &  \texttt{F1} \\
	\hline
	Lenovo &  \texttt{F1}, \texttt{F2} \\
	\hline
\end{tabular}
\end{table}
