\section{Algemene aanwijzingen}
\subsection{Typografische conventies}
Om te verduidelijken wat de bedoeling van een bepaald stukje tekst is, hanteren we typografi\"{e}en:
\begin{table}[H]
\begin{tabular}{| l | l |}
	\hline
	\textbf{Bold face} & voor invoer en optie namen van de Debian Installer\\
	\hline
	\texttt{Typewriter} & voor commando's voor de opdrachtregel en sneltoetsen\\
	\hline
\end{tabular}
\end{table}

\subsection{BIOS instellingen}
Om een installatie goed te kunnen uitvoeren is het meestal nodig de BIOS van de te installeren machine in te gaan. De toetsaanslagen om dit te bereiken verschillen per merk en model computer, maar in geval van twijfel kun je altijd de \texttt{Del}-toets proberen. Hieronder een overzicht per merk.
\begin{table}[H]
	\begin{tabular}{| l | l |}
	\hline 
	\textbf{Merk} & \textbf{Toetsen} \\
	\hline
	abit & \texttt{Del} \\
	\hline
	ASRock & \texttt{F2} \\
	\hline
	ASUS & \texttt{Del}, \texttt{Ins}, soms \texttt{F10} \\
	\hline
	BFG & \texttt{Del} \\
	\hline
	BIOSTAR & \texttt{Del} \\
	\hline
	Compaq & \texttt{F10}, \texttt{F1}, \texttt{F2} of \texttt{Delete}\\
	\hline
	Dell & \texttt{F2} \\
	\hline
	DFI & \texttt{Del} \\
	\hline
	ECS Elitegroup & \texttt{Del}, \texttt{F1} \\
	\hline
	EVGA & \texttt{Del} \\
	\hline
	Fujitsu & \texttt{F2} \\
	\hline
	Foxconn & \texttt{Del} \\
	\hline
	GIGABYTE & \texttt{Del} \\
	\hline
	Hewlett-Packard &  \texttt{F1}, \texttt{F10}, \texttt{F11} \\
	\hline
	IBM &  \texttt{F1} \\
	\hline
	Intel & \texttt{F2} \\
	\hline
	JetWay & \texttt{Del} \\
	\hline
	Lenovo &  \texttt{F1}, \texttt{F2} \\
	\hline
	Mach Speed & \texttt{Del} \\
	\hline
	MSI & \texttt{Del} \\
	\hline
	PCChips & \texttt{Del}, \texttt{F1} \\
	\hline
	SAPPHIRE & \texttt{Del} \\
	\hline
	Shuttle & \texttt{Del}, \texttt{Ctrl+Alt+Esc} \\
	\hline
	Soyo & \texttt{Del} \\
	\hline
	Super Micro & \texttt{Del} \\
	\hline
	TYAN & \texttt{Del}, \texttt{F4} \\
	\hline
	XFX & \texttt{Del} \\
	\hline
\end{tabular}
\end{table}
