\section{Handmatige afwerking}
\subsection{Geluid configuratie}

Om het geluidssysteem te initialiseren geeft je het volgende commando:

\begin{lstlisting}[language=bash]
alsactl init
\end{lstlisting}

Nu ga je het geluid testen met het volgende commando, wat je kunt (en moet) onderbreken met \textsc{Control-C}:

\begin{lstlisting}[language=bash]
speaker-test
\end{lstlisting}

Mocht het dan nog niet werken controleer dan de volumes met:

\begin{lstlisting}[language=bash]
alsamixer
\end{lstlisting}

en probeer \texttt{speaker-test} opnieuw.


\subsection{Graphische configuratie}

We kijken eerst of het al werkt met het commando:

\begin{lstlisting}[language=bash]
/etc/init.d/lightdm restart
\end{lstlisting}

Indien je een grapische omgeving ziet ben je klaar. Mocht dat niet zo zijn gebruik dan \textsc{Alt-Control-F2} om je weer als \textbf{root} te kunnen aanmelden en geef dan het volgende commando:


\begin{lstlisting}[language=bash]
xconfig.sh
\end{lstlisting}

\subsubsection{Nvida configuratie}

Kijk met 
\begin{lstlisting}[language=bash]
lspci | grep -i vga
\end{lstlisting}
of er een ondersteunde Nvidia kaart in de computer zit. Dit doe je door het model op te zoeken in \textit{supported devices} op de pagina \href{https://wiki.debian.org/NvidiaGraphicsDrivers}{Debian Wiki}. Dan weet je welke van de  \texttt{nvidiaXXX.sh} je moet aanroepen.

\subsection{Firmware installeren}
\label{ssec:Firmware}
Kijk met 
\begin{lstlisting}[language=bash]
dmesg | grep -i firmware | grep -i 'failed\|missing'
\end{lstlisting}
of er firmware ontbreekt op computer.
Indien dit commando uitvoer heeft mist de computer firmware. In de regel wordt de bestandsnaam van de ontbrekende firmware genoemd. Gebruik deze naam om met \texttt{apt-file search} het pakket op te zoeken dat dit bestand levert. Daarna installeer je dat pakket met \texttt{apt-get install}.

\subsection{Wireless configuratie}
Kijk met 
\begin{lstlisting}[language=bash]
iwconfig
\end{lstlisting}
of er draadloze interfaces zijn. Als deze er niet zijn ben je klaar. 

Maar is, zeg \texttt{wlan0}, een draadloze interface. Dan zet je de interface aan met 
\begin{lstlisting}[language=bash]
ifconfig wlan0 up
\end{lstlisting}
Als dit zonder uitvoer retourneert, heeft de draadloze interface geen firmware nodig.

Maar als er wel uitvoer is moet je de stappen uit \ref{ssec:Firmware} nogmaals uitvoeren.
\subsubsection{WICD configureren}
Je hebt nu de graphische omgeving nodig. Als je daar niet bent kun je er komen door \textsc{Alt-F7} in te drukken.
In het systeemvak vind je het pictogram van de WICD. Je klikt hierop en opent dan via het menu aan de rechterzijde de instellingen. 

Nu moeten we weten hoe de wireless interface heet in het systeem. Dit vind je uit door:
\begin{lstlisting}[language=bash]
iwconfig 2>/dev/null
\end{lstlisting}
Aan de linkerkant van de uitvoer staat nu de naam van de wireless interface, het zal \textbf{wlan0} of  \textbf{wlan1} zijn. Deze vul je in onder het kopje \textbf{Netwerkadapters} bij \textbf{Draadloze netwerkadapter}.
