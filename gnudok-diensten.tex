%%=====================================================================================
%%
%%       Filename:  gnudok-diensten.tex
%%
%%    Description:  Beschrijving van diensten van GNUDok
%%
%%        Version:  1.0
%%        Created:  11-02-15
%%       Revision:  none
%%
%%         Author:  Steven Speek 
%%   Organization:  
%%      Copyright:  
%%
%%          Notes: Not standalone 
%%                
%%=====================================================================================

\section{Inleiding}

Het GNUDok is de ICT-afdeling van stichting het \href{http://www.juttersdok.nl/}{Juttersdok} te Amsterdam. Wij bieden stagiaires de mogelijkheid met het vak kennis te maken zonder daarbij gesloten, of betaalde, programma's te gebruiken. Alles wat zij leren mogen ze overal over de wereld uitrollen zonder zich daarbij zorgen te hoeven maken over kosten van de software. Tevens bieden wij vrijwilligers die de vrije software gedachte een warm hart toedragen een plek om zich nuttig te maken voor de gebruikers van vrije software.

\section{Diensten}

Aan klanten van het Juttersdok bieden de volgende dingen aan:

\subsection{Installatie Debian}

Een complete Installatie van de degelijke linux distributie Debian.
Speerpunten hierbij zijn:

\begin{enumerate}
	\item \href{https://wiki.archlinux.org/index.php/Disk_encryption}{Disk versleuteling}
		\item geheel open source, dus geen spyware mogelijkheid
		\item \href{https://www.torproject.org}{Tor}, The Onion Router, zodat u echt anoniem het internet op kunt
\end{enumerate}


Verder geven wij advies over het gebruik van de \href{https://packages.debian.org/stable/}{programma's} die op linux kunnen worden ge\"{i}nstalleerd.

\subsection{Bloemlezing software}
Op linux is zo goed als alle software functionaliteit beschikbaar die op commerciele systemen beschikbaar is. Dit uiteraard zonder de haken en ogen die aan commerciele software kleven, zoals uitprobeer versies, spionage functionaliteit en bewuste onderwerping van de eindgebruiker. 

\subsubsection{Window managers}
Voor mensen die nog nooit het X-Windows hebben gewerkt is dit een nieuw begrip. Zowel Windows als MacOS hebben namelijk maar \'{e}\'{e}n windowmanager. 

De \href{https://en.wikipedia.org/wiki/Window_manager}{window manager} is verantwoordelijk voor het origaniseren van de geopende applicaties. Denk aan de \textsc{ALT-TAB} functionaliteit om van applicatie te wisselen. Maar ook het minimaliseren en maximaliseren van applicaties.

Linux kent:

\begin{enumerate}
	\item \href{https://www.kde.org/}{KDE}, een mooie rijke, maar ook zware window manager. Ze kan heel veel, en het ziet er mooi uit, maar is traag. Lijkt op wat windows en MacOS doen.
	\item \href{http://www.gnome.org/}{GNOME-shell} lijkt op KDE, maar is wat lichter en effic\"{i}enter
	\item \href{http://lxde.org/}{LXDE} een soberdere snelle window manager, de standaard voor GNUDok installaties.
	\item \href{http://i3wm.org/}{i3}, \href{http://awesome.naquadah.org/}{awesomewm} en \href{http://xmonad.org/}{XMonad}: supersnelle \href{https://en.wikipedia.org/wiki/Tiling_window_manager}{tiling window managers} voor power users.
\end{enumerate}

\subsubsection{Browsers}
Om het world wide web te bezoeken heeft linux vele mogelijkheden:

\begin{enumerate}
	\item \href{https://www.mozilla.org/nl/firefox/new/}{Firefox}
	\item \href{http://www.chromium.org/}{Chromium}
	\item \href{http://portix.bitbucket.org/dwb/}{Dynamic web browser} (dwb), een slanke snelle webkit-gebaseerde browser.
\end{enumerate}

