%%=====================================================================================
%%
%%       Filename:  gnudok-diensten.tex
%%
%%    Description:  Beschrijving van diensten van GNUDok
%%
%%        Version:  1.0
%%        Created:  11-02-15
%%       Revision:  none
%%
%%         Author:  Steven Speek 
%%   Organization:  
%%      Copyright:  
%%
%%          Notes: Not standalone 
%%                
%%=====================================================================================

\section{Inleiding}

Het GNUDok is de ICT-afdeling van stichting het \href{http://www.juttersdok.nl/}{Juttersdok} te Amsterdam. Wij bieden stagiaires de mogelijkheid met het vak kennis te maken zonder daarbij gesloten, of betaalde, programma's te gebruiken. Alles wat zij leren mogen ze overal over de wereld uitrollen zonder zich daarbij zorgen te hoeven maken over kosten van de software. Tevens bieden wij vrijwilligers die de \href{http://www.fsf.org/}{vrije software gedachte} een warm hart toedragen de ruimte om zich nuttig te maken voor de gebruikers van vrije software.

\section{Diensten}

Aan klanten van het Juttersdok biedt GNUDok de volgende diensten aan:

\subsection{Installatie Debian}

Een complete installatie van de degelijke linux distributie \href{https://www.debian.org/}{Debian}.
Speerpunten hierbij zijn:

\begin{enumerate}
	\item \href{https://wiki.archlinux.org/index.php/Disk_encryption}{Disk versleuteling}.
		\item Geheel opgebouwd uit vrije software; u kunt dus zelf aan de hand van de broncode nagaan dat er geen achterdeuren of spionage functionaliteiten in de programmatuur zitten.
		\item \href{https://www.torproject.org}{Tor}, The Onion Router, zodat u echt anoniem het internet op kunt.
\end{enumerate}


Verder geven wij advies over het gebruik van de \href{https://packages.debian.org/stable/}{programma's} die op linux kunnen worden ge\"{i}nstalleerd.

\subsection{Bloemlezing software}
Op linux is zo goed als alle software functionaliteit beschikbaar die op commerciele systemen beschikbaar is. Dit uiteraard zonder de haken en ogen die aan commerciele software kleven, zoals uitprobeer versies, spionage functionaliteit of bewuste onderwerping van de eindgebruiker. 

Hier beneden worden van een aantal categori\"{e}en applicaties de populaire programma's besproken. Dit is nooit een complete lijst, maar het beste wat er op dat gebied is gemaakt.

\subsubsection{Browsers}
Om het world wide web te bezoeken biedt linux:

\begin{enumerate}
	\item \href{https://www.mozilla.org/nl/firefox/new/}{Firefox}.
	\item \href{http://www.chromium.org/}{Chromium}.
	\item \href{http://portix.bitbucket.org/dwb/}{Dynamic web browser} (dwb), een slanke snelle webkit-gebaseerde browser.
\end{enumerate}

\subsubsection{Foto organisatie en bewerking}
Om uw foto's te organiseren en bewerken zijn er de volgende mogelijkheden:

\begin{enumerate}
	\item \href{https://wiki.gnome.org/Apps/Shotwell}{Shotwell} foto- en filmbeheer.
	\item \href{https://www.digikam.org/}{Digikam}  foto- en filmbeheer.
	\item \href{http://www.gimp.org/}{The GIMP} voor foto bewerking.
\end{enumerate}

\subsubsection{Film en video spelers}
Uw gedownloade of aangekochte films en video kunt bekijken met:
\begin{enumerate}
	\item \href{http://www.mplayer.org/}{MPlayer}.
	\item \href{http://www.videolan.org/vlc/}{VLC}, de Video LAN Player.
\end{enumerate}

\subsubsection{Muziek spelers}
\begin{enumerate}
	\item \href{https://wiki.gnome.org/Apps/Rhythmbox}{Rhythmbox} is een complete muziekspeler.
	\item \href{http://mixxx.org/}{Mixxx} is een heuse DJ-applicatie die zelfs automatisch kan mixen.
\end{enumerate}

\subsubsection{E-book beheer}
Om uw e-books beheren en te delen met anderen heeft linux \href{http://calibre-ebook.com/}{Calibre}.

\subsubsection{Multimedia totaal oplossingen}
\begin{enumerate}
	\item \href{http://popcorntime.io/}{Popcorntime}, een gratis Netflix gebaseerd op het torrent netwerk.
		\item \href{http://kodi.tv/}{Kodi}, voorheen XBMC.
\end{enumerate}

\subsubsection{Bestanden delen}
\begin{enumerate}
	\item \href{http://www.ktorrent.org/}{KTorrent} is een complete torrent client met vele mogelijkheden.
	\item \href{https://www.transmissionbt.com/}{Transmission} is een simpele torrent client.
\end{enumerate}

\subsubsection{Kantoor software}
Om documenten, spreadsheets en presentaties te maken heeft linux \href{https://www.libreoffice.org/}{LibreOffice}.

\subsubsection{Educatieve software}
\begin{enumerate}
	\item \href{http://gcompris.net/index-en.html}{GCompris} voor de kleintjes.
	\item \href{http://tux4kids.alioth.debian.org/}{Tux4Kids} met daarin TuxMath, TuxPaint en TuxTyping.
	\item \href{https://edu.kde.org/kstars/}{KStars} om de sterrehemel te tonen zoals ze er nu uitziet.
	\item \href{https://edu.kde.org/marble/} Marble, een vrije implementatie van Google Earth.
\end{enumerate}

\subsubsection{Teken programma's}

\begin{enumerate}
	\item \href{https://inkscape.org/nl/}{Inkscape} om vector tekeningen te maken.
	\item \href{https://wiki.gnome.org/Apps/Dia}{Dia} een open source Visio kloon. 
\end{enumerate}

\subsubsection{Video bewerking}
Video's kunt u bewerken met \href{https://kdenlive.org/}{Kdenlive}. U kunt er geluid bij mixen en effecten gebruiken.

\subsubsection{Geluid bewerking en componeren}
\begin{enumerate}
	\item \href{http://www.rosegardenmusic.com/}{Rosegarden} is een muziek componeer toepassing.
	\item \href{http://audacity.sourceforge.net/}{Audacity} is een uitstekend programma om geluid te remasteren.
\end{enumerate}


\subsubsection{Spellen}
Er zijn de gebruikelijke bord- en kaartspelen, maar ook heuse arcade-spellen. Voorbeelden:
\begin{enumerate}
	\item \href{http://pychess.org/}{Pychess} is een zeer compleet schaakprogramma.
	\item \href{http://www.minetest.net/}{Minetest} is een vrije implementatie van Minecraft.
\end{enumerate}

\subsubsection{Window managers}
Voor mensen die nog nooit het X-Windows hebben gewerkt is dit een nieuw begrip. Zowel Windows als MacOS hebben namelijk maar \'{e}\'{e}n windowmanager. 

De \href{https://en.wikipedia.org/wiki/Window_manager}{window manager} is verantwoordelijk voor het origaniseren van de geopende applicaties. Denk aan de \textsc{ALT-TAB} functionaliteit om van applicatie te wisselen. Maar ook het minimaliseren en maximaliseren van applicaties.

Linux kent:

\begin{enumerate}
	\item \href{https://www.kde.org/}{KDE}, een mooie rijke, maar ook zware window manager. Ze kan heel veel, en het ziet er mooi uit, maar is traag. Lijkt op wat windows en MacOS doen.
	\item \href{http://www.gnome.org/}{GNOME-shell} lijkt op KDE, maar is wat lichter en effic\"{i}enter
	\item \href{http://lxde.org/}{LXDE} een soberdere snelle window manager, de standaard voor GNUDok installaties.
	\item \href{http://i3wm.org/}{i3}, \href{http://awesome.naquadah.org/}{awesomewm} en \href{http://xmonad.org/}{XMonad}: supersnelle \href{https://en.wikipedia.org/wiki/Tiling_window_manager}{tiling window managers} voor power users.
\end{enumerate}

\subsubsection{Programmeren}
Voor de mogelijkheden om zelf programma's te maken op linux verwijs ik naar \href{https://www.debian.org/doc/manuals/debian-reference/ch12.en.html}{Debian Reference, Hoofdstuk Programmeren}.

\subsubsection{Wetenschappelijke programma's}
Debian linux is de beste distributie om wetenschappelijk onderzoek te ondersteunen. Een paar voorbeelden:

\begin{enumerate}
	\item \LaTeX type setting voor publicaties.
	\item \href{https://coq.inria.fr/}{Coq} theorem prover.
\end{enumerate}

