\section{FAI server instellen}
Om te kunnen bestaan heeft de nieuw te installeren machine een unieke hostnaam nodig van de vorm \texttt{debianNN}. Deze zal in \texttt{/etc/hosts} moeten worden opgenomen. Vervolgens moet het MAC-adres van de computer hieraan worden gekoppeld. Alle commando's die je in de volgende drie secties ziet staan, dienen als root op een faiserver uitgevoerd te worden.
\subsection{Hosts bestand}
Voor de installatie van machine voor de verkoop gebruiken we hostnamen zoals \texttt{debianNN} (dit is bijvoorbeeld \texttt{debian02}).
In \texttt{/etc/hosts} moet een hostnaam en een IP-adres worden gedefinieerd (zie eventueel \texttt{man hosts}).
Op de hosts file met vim:
\begin{lstlisting}[language=bash]
vim /etc/hosts
\end{lstlisting}
Enkele tips hierbij zijn om de laatste regel van de vorm:
\begin{lstlisting}[language=bash]
10.0.x0.NNN			debianMM
\end{lstlisting}
te kopieren met \texttt{yy} en daaronder te plakken met \texttt{p}. Je navigeert tot onder het laatste getal van het ip-adres (in normal mode) met \texttt{E} (je kunt terug met \texttt{b}). Dan geef je \texttt{Control-a} om het getal met \'{e}\'{e}n op te hogen. Dit doe je ook het \texttt{debianMM}. Dan sla het bestand op het \texttt{ZZ}. Merk op dat je helemaal niet in insert mode hoeft te gaan.
\subsection{DHCP-server instellen}
Met het speciaal daarvoor bestemde fai-util \texttt{dhcp-edit} voeg je deze in vorige sectie gekozen hostnaam en mac-adres toe.
\begin{lstlisting}[language=bash]
dhcp-edit debianNN 00:11:22:33:44:55
\end{lstlisting}
\subsection{PXE instellen}
Bij deze stap moet je weten of je een 32-bit of een 64-bit computer wilt installeren.
Voor 64-bit luidt commando :
\begin{lstlisting}[language=bash]
pxe.sh debianNN
\end{lstlisting}
Voor 32-bit is het:
\begin{lstlisting}[language=bash]
pxe-i386.sh debianNN
\end{lstlisting}

Nu zou je je te installeren machine kunnen laten booten van haar netwerkkaart, al dan niet via \texttt{gPXE 1.0.1}-cdrom,  en de automatische installatie zou moeten beginnen.

