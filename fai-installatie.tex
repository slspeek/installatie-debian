\section{Voorbereiding}
\subsection{MAC adres achterhalen}
Laat de machine die je wilt installeren opstarten en ga de BIOS in. In de BIOS schakel je boot ROM faciliteiten in. En configureer je de netwerkkaart als eerste boot-optie. Dan laat je de netwerkboot lopen en schrijft van de scherm het MAC-adres over. Dit zijn zes groepjes van twee cijfers en letters met dubbele punten ertussen. 

\section{FAI server instellen}
Om te kunnen bestaan heeft de nieuw te installeren machine een unieke hostnaam nodig van de vorm \texttt{debianNN}. Deze zal in \texttt{/etc/hosts} moeten worden opgenomen. Vervolgens moet het MAC-adres van de computer hieraan worden gekoppeld. Alle commando's die je in de volgende drie secties ziet staan, dienen als root op de faiserver uitgevoerd te worden.
\subsection{Hosts bestand}
Voor de installatie van machine voor de verkoop gebruiken we hostnamen zoals \texttt{debianNN} (dit is bijvoorbeeld \texttt{debian02}).
In \texttt{/etc/hosts} moet een hostnaam en een IP-adres worden gedefinieerd (zie eventueel \texttt{man hosts}). Met het volgende commando bepaal je de hoogst in gebruik zijnde \texttt{debianNN} en ip-adres.  
\begin{lstlisting}[language=bash]
grep debian /etc/hosts|sort -r|head -1
\end{lstlisting}
Je gaat precies \'{e}\'{e}n hoger zitten. Deze voeg je toe aan \texttt{/etc/hosts}, met het commando
\begin{lstlisting}[language=bash]
vim /etc/hosts
\end{lstlisting}
\subsection{DHCP-server instellen}
Met het speciaal daarvoor bestemde fai-util \texttt{dhcp-edit} voeg je deze in vorige sectie gekozen hostnaam en mac-adres toe.
\begin{lstlisting}[language=bash]
dhcp-edit debianNN 00:11:22:33:44:55
\end{lstlisting}
Dan moet de DHCP-server herstart worden met:

\begin{lstlisting}[language=bash]
/etc/init.d/isc-dhcp-server restart
\end{lstlisting}


\subsection{PXE instellen}
Bij deze stap moet je weten of je een 32-bit of een 64-bit computer wilt installeren.
Voor 64-bit luidt commando:
\begin{lstlisting}[language=bash]
pxe.sh debianNN
\end{lstlisting}
Voor 32-bit is het:
\begin{lstlisting}[language=bash]
pxe-i386.sh debianNN
\end{lstlisting}

Nu zou je je te installeren machine kunnen laten booten van haar netwerkkaart en de automatische installatie zou moeten beginnen.

