\section{Algemene aanwijzingen}
\emph{Zeer belangrijk:} Iedere vraag die door het installatie programma aan jou wordt gesteld en waarover hieronder niets staat vermeldt beantwoordt je met de \texttt{enter}-toets.

Als de debian installer naar een nieuw scherm gaat, begint er in deze handleiding een nieuwe paragraaf.
\subsection{Typografische conventies}
Om te verduidelijken wat de bedoeling van een bepaald stukje tekst is, hanteren we typografi\"{e}en:
\begin{table}[H]
\begin{tabular}{| l | l |}
	\hline
	\textbf{Bold face} & voor invoer en optie namen van de Debian Installer\\
	\hline
	\texttt{Typewriter} & voor commando's voor de opdrachtregel en sneltoetsen\\
	\hline
\end{tabular}
\end{table}

\section{BIOS instellingen}
Om een installatie goed te kunnen uitvoeren is het meestal nodig de BIOS van de te installeren machine in te gaan. De toetsaanslagen om dit te bereiken verschillen per merk en model computer. Hieronder een overzicht per merk.
\begin{table}[H]
	\begin{tabular}{| l | l |}
	\hline 
	\textbf{Merk} & \textbf{Toetsen} \\
	\hline
	Compaq & \texttt{F10}, \texttt{F1}, \texttt{F2} of \texttt{Delete}\\
	\hline
	Dell & \texttt{F2} \\
	\hline
	Fujitsu & \texttt{F2} \\
	\hline
	Hewlet Packert &  \texttt{F1}, \texttt{F10}, \texttt{F11} \\
	\hline
	IBM &  \texttt{F1} \\
	\hline
	Lenovo &  \texttt{F1}, \texttt{F2} \\
	\hline
\end{tabular}
\end{table}
\subsection{Netwerk boot}
Moderne machines kunnen opstarten (booten) van de netwerkkaart, de toetsen verschillen per merk. Sommige machine kunnen direct van de netwerkkaart booten, bij andere moet je dat via het boot-menu doen. Hieronder een overzicht:
\begin{table}[H]
	\begin{tabular}{| l | l |  l |}
	\hline 
	\textbf{Merk} & \textbf{Netwerk boot toets} & \textbf{Boot menu toets}\\
	\hline
	Compaq &  & \texttt{F12}\\
	\hline
	Dell &  & \texttt{F12} \\
	\hline
	Fujitsu &  & \\
	\hline
	Hewlet Packert &   & \\
	\hline
	IBM &   & \\
	\hline
	Lenovo &  &  \\
	\hline
\end{tabular}
\end{table}

\section{Floppy station onklaar maken}
Floppy stations zijn een erfenis uit het verleden. Schakel de floppy ondersteuning in de BIOS uit. En maak de kabels verbonden met het floppy-station los.


