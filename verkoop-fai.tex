\section{Afwerking voor verkoop}
Nu volgen instructies voor het afmaken van de machine voordat ze in de verkoop kan.
\subsection{Verkoopblad maken}
Log in op een werkstation op het netwerk. Je vindt op \texttt{/home/netwerk/GNUDokComputerPrijzen.ots} het sjabloon weermee je het verkoopblad mee moet maken. Je vult op het tabblad `Spec' de volgende zaken in:
\begin{itemize}
	\item Processor snelheid, uit te vinden met \texttt{lscpu}.
	\item De registerbreedte; 32 of 64 bit, ook uit te vinden met \texttt{lscpu}.
	\item Of het een dualcore is. Met \texttt{lscpu} kijk je bij `Core(s) per socket', als dat 2 is heb je een dualcore voor je.
	\item Of de machine een HyperThreader is. Met \texttt{lscpu} kijk je bij `Threads(s) per core', als dat 2 is heb je een HyperThreader voor je. 
	\item Of er een wireless interface in zit. Met \texttt{iwconfig} als \texttt{root}.
	\item De hoeveelheid RAM in megabites. Dit doe je met \texttt{free}.
	\item Het aantal giga bytes van de harddisk. Dit vindt je met \texttt{dmesg|grep GB}.
	\item Of de graphische kaart versneld is. Dit vindt je uit met \texttt{lsmod|grep nvidia}.
\end{itemize}

Als je klaar bent print je het `Voorblad' tabje uit het document. Dit stop je in een plastic insteekhoes en bevestig je met brede plastic tape aan de computer.
\subsection{Reinigen van de hardware}
\subsubsection{Verwijderen van de Microsoft licensie sticker}
Je legt een natte vaatdoek met een beetje zeep op de licensie sticker en laat deze tien minuten weken, daarna verwijder je de sticker. Je nu de belangrijkste zuivering acher de rug hebt ga je verder met het schoonmaken van de kast.
\subsubsection{Reinigen kast}
Maak de kast schoon met een licht vochtige doek. Dompel hiertoe de doek in een sopje van water met een scheutje allesreiginer, en wring het hierna uit. Verwijder met de doek stof en andere ongerechtigheden van de systeemkast.
