\documentclass[12pt,a4paper]{article}
\usepackage[utf8]{inputenc}
\usepackage{amsmath}
\usepackage{amsfonts}
\usepackage{amssymb}
\usepackage{graphicx}
\begin{document}
\graphicspath{ {./images/} }
\author{Steven L. Speek}
\title{GNUAmsterdam Debian Wheezy Installatie}
\date{\today}
\maketitle
\abstract{Nederlandse handleiding voor een Debian desktop installatie.}
\section{Algemene aanwijzingen}
\emph{Zeer belangrijk:} Iedere vraag die door het installatie programma aan jou wordt gesteld en waarover hieronder niets staat vermeldt beantwoordt je met de enter-toets.
\section{Floppy station onklaar maken}
Floppy stations zijn een erfenis uit het verleden. Schakel de floppy ondersteuning in de BIOS uit. En maak de kabels verbonden met het floppy-station los.
\section{Installatie basissysteem}
Boot van een debian installer CDROM (Op te halen op adres: www.debian.org/distrib/, kies “klein installatie-image”).
Je komt dan in een menu. 
Je accepteert de eerste optie {\bf Install } met 'enter'. Kies {\bf Dutch}, pijltje omhoog en dan enter. Regio/Land: {\bf Nederland} en druk enter. Kies {\bf American English}, druk op enter.
\end{document}