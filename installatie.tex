\documentclass[12pt,a4paper]{article}
\usepackage[utf8]{inputenc}
\usepackage{amsmath}
\usepackage{amsfonts}
\usepackage{amssymb}
\usepackage{graphicx}
\usepackage{listings}
\begin{document}
\graphicspath{ {./images/} }
\lstset{language=bash}      
\author{Steven L. Speek}
\title{GNUAmsterdam Debian Wheezy Installatie}
\date{\today}
\maketitle
\abstract{Nederlandse handleiding voor een Debian desktop installatie.}
\section{Algemene aanwijzingen}
\emph{Zeer belangrijk:} Iedere vraag die door het installatie programma aan jou wordt gesteld en waarover hieronder niets staat vermeldt beantwoordt je met de enter-toets.
\section{Floppy station onklaar maken}
Floppy stations zijn een erfenis uit het verleden. Schakel de floppy ondersteuning in de BIOS uit. En maak de kabels verbonden met het floppy-station los.
\section{CDROM ophalen}
Op www.debian.org/distrib/, kies je een klein installatie-image voor jouw architectuur (meestal amd64, soms nog i686).
Brandt dit op een CDROM, dit zullen wij de installatie CDROM noemen.
\section{Installatie basissysteem}
Boot van de installatie CDROM. Je komt dan in een menu.
Je accepteert de eerste optie \textbf{Install} met enter.
Op het volgende scherm kun je de taal kiezen; kies \textbf{Dutch} (pijltje omhoog en dan enter).
Regio/Land: \textbf{Nederland} en druk enter.
Voor de toetsenbordindeling kies je \textbf{American English}, druk op enter.
Vul \textbf{debian} in als computernaam en druk op enter.
En vul \textbf{lan} in bij domeinnaam en druk op enter.
Nu moeten het root wachtwoord worden gezet.
Vul als root wachtwoord in \textbf{root} (bevestigen met enter), en doe dit nogmaals ter controle.
Je vult \textbf{Tux} in als volledige naam van de nieuwe aardse gebruiker.
De debian installer suggereert dan \textbf{tux} als gebruikersnaam.
Dat accepteer je met enter.
Als wachtwoord vul je \textbf{tux} in. Dit moet nog een keer worden bevestigd.
\section{Schijfindeling voor de verkoop}
Kopen, hoor!
\begin{lstlisting}[language=bash]
$lsof -i :80
\end{lstlisting}
Shows open files.
\end{document}
